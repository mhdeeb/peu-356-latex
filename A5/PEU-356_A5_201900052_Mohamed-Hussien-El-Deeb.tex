\documentclass[12pt]{article}
\usepackage[svgnames,x11names,table]{xcolor}
\usepackage{hyperref}
\usepackage{graphicx}
\usepackage{parskip}
\usepackage{float}
\usepackage{amsmath}
\usepackage{esint}
\usepackage{amssymb}
\usepackage{enumitem}
\usepackage[thicklines]{cancel}

\hypersetup{
    colorlinks,
    citecolor=blue,
    filecolor=black,
    linkcolor=black,
    urlcolor=RoyalBlue4,
}

\title{PEU 356 Assignment 5}
\author{Mohamed Hussien El-Deeb (201900052)}
\date{\today}

\DeclareMathOperator{\sech}{sech}
\DeclareMathOperator{\csch}{csch}

\begin{document}

\maketitle
\tableofcontents
\hypersetup{linkcolor=RoyalBlue4}

\newpage
\section{5.1.11}

\subsection{Problem}

Using conventional vector notation, evaluate \(\sum_j\left|\hat{\mathbf{e}}_j\right\rangle\left\langle\hat{\mathbf{e}}_j \mid \mathbf{a}\right\rangle \), where \(\mathbf{a}\) is an arbitrary vector in the space spanned by the \(\hat{\mathbf{e}}_j\).

\subsection{Solution}

\[
    \sum_j\left|\hat{\mathbf{e}}_j\right\rangle\left\langle\hat{\mathbf{e}}_j \mid \mathbf{a}\right\rangle=\sum_j\left|\hat{\mathbf{e}}_j\right\rangle\left(\hat{\mathbf{e}}_j \cdot \mathbf{a}\right)=\sum_j\left(\hat{\mathbf{e}}_j \cdot \mathbf{a}\right)\hat{\mathbf{e}}_j
\]

This is the projection of \(\mathbf{a}\) onto the basis vectors \(\hat{\mathbf{e}}_j\).

\newpage
\section{5.1.12}

\subsection{Problem}

Letting \(\mathbf{a}=a_1 \hat{\mathbf{e}}_1+a_2 \hat{\mathbf{e}}_2\) and \(\mathbf{b}=b_1 \hat{\mathbf{e}}_1+b_2 \hat{\mathbf{e}}_2\) be vectors in \(\mathbb{R}^2\), for what values of \(k\), if any, is

\[
    \langle\mathbf{a} \mid \mathbf{b}\rangle=a_1 b_1-a_1 b_2-a_2 b_1+k a_2 b_2
\]

a valid definition of a scalar product?

\subsection{Solution}

\[
    \langle\mathbf{a} \mid \mathbf{a}\rangle = {(a_1 - a_2)}^2 + (k - 1) a_2^2 \geq 0
\]

For this to be true, \(k \geq 1\).


\newpage
\section{5.2.2}

\subsection{Problem}

Apply the Gram-Schmidt procedure to form the first three Laguerre polynomials:
\[
    u_n(x)=x^n, \quad n=0,1,2, \ldots, \quad 0 \leq x<\infty, \quad w(x)=e^{-x} .
\]

The conventional normalization is

\[
    \int_0^{\infty} L_m(x) L_n(x) e^{-x} d x=\delta_{m n} .
\]

ANS. \(\quad L_0=1, \quad L_1=(1-x), \quad L_2=\frac{2-4 x+x^2}{2}\).

\subsection{Solution}

\[
    L_n = x^n - \sum_{k=0}^{n-1} \left\langle x^n \mid L_k\right\rangle \tilde{L_k}
\]

\[
    \tilde{L_k} = \frac{L_k}{\left\langle L_k \mid L_k\right\rangle}
\]

\[
    \hat{L_k} = \frac{L_k}{\sqrt{\left\langle L_k \mid L_k\right\rangle}}
\]

\[
    L_0 = 1
\]

\[
    \left\langle 1 \mid 1\right\rangle = \int_0^{\infty} e^{-x} d x = 1
\]

\[
    \tilde{L_0} = \frac{L_0}{\left\langle L_0 \mid L_0\right\rangle}  = \frac{1}{\left\langle 1 \mid 1\right\rangle} = 1
\]

\[
    \left\langle x \mid 1\right\rangle = \int_0^{\infty} x e^{-x} d x = 1
\]

\[
    L_1 = x - \left\langle x \mid 1\right\rangle 1 = x - 1
\]

\[
    \left\langle x - 1 \mid x - 1\right\rangle = \int_0^{\infty} {(x - 1)}^2 e^{-x} d x = 1
\]

\[
    \tilde{L_1} = \hat{L_1} = x - 1
\]

\[
    \left\langle x^2 \mid 1\right\rangle = \int_0^{\infty} x^2 e^{-x} d x = 2
\]

\[
    \left\langle x^2 \mid x-1\right\rangle = \int_0^{\infty} x^2 (x-1) e^{-x} d x = 4
\]

\[
    L_2 = x^2 - 2 - 4 (x-1) = x^2 - 4 x + 2
\]

\[
    \left\langle L_2 \mid L_2\right\rangle = \int_0^{\infty} {(x^2 - 4 x + 2)}^2 e^{-x} d x = 4
\]

\[
    \hat{L_2} = \frac{x^2 - 4 x + 2}{2}
\]

\newpage
\section{5.2.4}

\subsection{Problem}

Using the Gram-Schmidt orthogonalization procedure, construct the lowest three Hermite polynomials:

\[
    u_n(x)=x^n, \quad n=0,1,2, \ldots, \quad-\infty<x<\infty, \quad w(x)=e^{-x^2} .
\]

For this set of polynomials the usual normalization is

\[
    \begin{aligned}
        \int_{-\infty}^{\infty} H_m(x) H_n(x) w(x) d x= & \delta_{m n} 2^m m!\pi^{1 / 2}            \\
        \text { ANS. }                                  & H_0=1, \quad H_1=2 x, \quad H_2=4 x^2-2 .
    \end{aligned}
\]

\subsection{Solution}

\[
    H_n = x^n - \sum_{k=0}^{n-1} \left\langle x^n \mid H_k\right\rangle \tilde{H_k}
\]

\[
    \tilde{H_k} = \frac{H_k}{\left\langle H_k \mid H_k\right\rangle}
\]

\[
    \hat{H_k} = \sqrt{\frac{2^k k! \sqrt{\pi}}{\left\langle H_k \mid H_k\right\rangle}}H_k
\]

\[
    H_0 = 1
\]

\[
    \left\langle H_0 \mid H_0\right\rangle = \int_{-\infty}^{\infty} e^{-x^2} d x = \sqrt{\pi}
\]

\[
    \tilde{H_0} = \frac{H_0}{\left\langle H_0 \mid H_0\right\rangle}  = \frac{1}{\left\langle 1 \mid 1\right\rangle} = \frac{1}{\sqrt{\pi}}
\]

\[
    \hat{H_0} = 1
\]

\[
    H_0 \equiv \hat{H_0}
\]

\[
    \left\langle x \mid H_0\right\rangle = \int_{-\infty}^{\infty} x e^{-x^2} d x = 0
\]

\[
    H_1 = x - \left\langle x \mid H_0\right\rangle \tilde{H_0} = x
\]

\[
    \left\langle H_1 \mid H_1\right\rangle = \int_{-\infty}^{\infty} x^2 e^{-x^2} d x = \frac{\sqrt{\pi}}{2}
\]

\[
    \tilde{H_1} = \frac{2 x}{\sqrt{\pi}}
\]

\[
    \hat{H_1} = 2 x
\]

\[
    H_1 \equiv \hat{H_1}
\]

\[
    \left\langle x^2 \mid H_0\right\rangle = \int_{-\infty}^{\infty} x^2 e^{-x^2} d x = \frac{\sqrt{\pi}}{2}
\]

\[
    \left\langle x^2 \mid H_1\right\rangle = 2 \int_{-\infty}^{\infty} x^3 e^{-x^2} d x = 0
\]

\[
    H_2 = x^2 - \left\langle x^2 \mid H_0\right\rangle \tilde{H_0} - \left\langle x^2 \mid H_1\right\rangle \tilde{H_1} = x^2 - \frac{1}{2}
\]

\[
    \left\langle H_2 \mid H_2\right\rangle = \int_{-\infty}^{\infty} {(x^2 - \frac{1}{2})}^2 e^{-x^2} d x = \frac{\sqrt{\pi}}{2}
\]

\[
    \hat{H_2} = 4 x^2 - 2
\]

\newpage
\section{5.3.1}

\subsection{Problem}

Show (without introducing matrix representations) that the adjoint of the adjoint of an operator restores the original operator, i.e., that \({\left(A^{\dagger}\right)}^{\dagger}=A\).

\subsection{Solution}

\[
    \left\langle\psi\left|A\right| \phi\right\rangle
    = \left\langle A^{\dagger} \psi \mid \phi\right\rangle
    = \left\langle\phi\left|A^{\dagger}\right| \psi\right\rangle^{*}
    = \left\langle {\left(A^{\dagger}\right) }^{\dagger} \psi \mid \phi\right\rangle^{*}
    = \left\langle\psi\left|{\left(A^{\dagger}\right) }^{\dagger}\right| \phi\right\rangle
\]

\newpage
\section{5.3.2}

\subsection{Problem}

\(U\) and \(V\) are two arbitrary operators. Without introducing matrix representations of these operators, show that

\[
    {(U V)}^{\dagger}=V^{\dagger} U^{\dagger} .
\]

Note the resemblance to adjoint matrices.

\subsection{Solution}

\[
    \left\langle\psi\left|U V\right| \phi\right\rangle
    = \left\langle U^{\dagger} \psi \left|V\right| \phi\right\rangle
    = \left\langle V^{\dagger} U^{\dagger} \psi \mid \phi\right\rangle
\]

\[
    \left\langle\psi\left|U V\right| \phi\right\rangle
    = \left\langle {\left(U V\right) }^{\dagger} \psi \mid \phi\right\rangle
\]

\[
    \left\langle {\left(U V\right) }^{\dagger} \psi \mid \phi\right\rangle
    = \left\langle V^{\dagger} U^{\dagger} \psi \mid \phi\right\rangle
\]

\[
    {\left(U V\right) }^{\dagger} = V^{\dagger} U^{\dagger}
\]

\newpage
\section{5.4.4}

\subsection{Problem}

The operator \(\mathcal{L}\) is Hermitian. Show that \(\left\langle\mathcal{L}^2\right\rangle \geq 0\), meaning that for all \(\psi \) in the space in which \(\mathcal{L}\) is defined, \(\left\langle\psi\left|\mathcal{L}^2\right| \psi\right\rangle \geq 0\).

\subsection{Solution}

\[
    \left\langle\psi\left|\mathcal{L}^2\right| \psi\right\rangle
    = \left\langle\psi\left|\mathcal{L} \mathcal{L}\right| \psi\right\rangle
    = \left\langle\mathcal{L}^\dagger \psi |\mathcal{L} \psi\right\rangle
    = \left\langle\mathcal{L} \psi |\mathcal{L} \psi\right\rangle
\]

By definition of inner product,

\[
    \left\langle\mathcal{L} \psi |\mathcal{L} \psi\right\rangle \geq 0
\]

\newpage
\bibliographystyle{plain}
\bibliography{references}
\nocite{arfken2013mathematical}
\nocite{El-Deeb_PEU-356_Assignments}

\end{document}