\documentclass[12pt]{article}
\usepackage[svgnames,x11names,table]{xcolor}
\usepackage{hyperref}
\usepackage{graphicx}
\usepackage{parskip}
\usepackage{float}
\usepackage{amsmath}
\usepackage{esint}
\usepackage{amssymb}
\usepackage{enumitem}
\usepackage[thicklines]{cancel}

\hypersetup{
    colorlinks,
    citecolor=blue,
    filecolor=black,
    linkcolor=black,
    urlcolor=RoyalBlue4,
}

\title{PEU 356 Assignment 3}
\author{Mohamed Hussien El-Deeb (201900052)}
\date{\today}

\DeclareMathOperator{\sech}{sech}
\DeclareMathOperator{\csch}{csch}

\begin{document}

\maketitle
\tableofcontents
\hypersetup{linkcolor=RoyalBlue4}

\newpage
\section{4.1.10}

\subsection{Problem}

The double summation \( K_{ij} A^i B^j \) is invariant for any two vectors \( A^i \) and \( B^j \). Prove that \( K_{ij} \) is a second-rank tensor.

\textit{Note.} In the form \( ds^2 \) (invariant) \( = g_{ij} dx^i dx^j \), this result shows that the matrix \( g_{ij} \) is a tensor.

\subsection{Solution}

\newpage
\section{4.2.2}

\subsection{Problem}

Show that the vector product is unique to 3-D space, that is, only in three dimensions can
we establish a one-to-one correspondence between the components of an antisymmetric
tensor (second-rank) and the components of a vector.

\subsection{Solution}

\newpage
\section{4.2.4(a)}

\subsection{Problem}

Verify that the following fourth-rank tensors is isotropic, that is, that it has the
same form independent of any rotation of the coordinate systems.

\[
    A^{ik}_{jl} = \delta^i_j \delta^k_l
\]

\subsection{Solution}

\newpage
\section{4.2.6}

\subsection{Problem}

Represent \(\varepsilon_{ij}\) by a 2 \(\times \) 2 matrix, and using the 2 \(\times \) 2 rotation matrix

\[
    \begin{pmatrix}
        \cos \varphi  & \sin \varphi \\
        -\sin \varphi & \cos \varphi
    \end{pmatrix}
\]

Show that \(\varepsilon_{ij}\) is invariant under orthogonal similarity transformations.

\subsection{Solution}

\newpage
\section{4.2.7}

\subsection{Problem}

Given \(A_k = \frac{1}{2}\varepsilon_{ijk}B^{ij}\) with \(B^{ij} = -B^{ji}\) antisymmetric, show that,

\[
    B^{mn} = \varepsilon_{mnk}A_k
\]

\subsection{Solution}

\newpage
\section{4.3.1}

\subsection{Problem}

For the special case of 3-D space (\(\varepsilon_1, \varepsilon_2, \varepsilon_3\) defining a right-handed coordinate system,
not necessarily orthogonal), show that

\[
    \varepsilon^i = \frac{\varepsilon_j \times \varepsilon_k}{\varepsilon_j \times \varepsilon_k \cdot \varepsilon_i}, \quad i = 1, 2, 3 \text{ and cyclic permutations. }
\]

Note. These contravariant basis vectors \(\varepsilon^i\) define the reciprocal lattice space of
Example 3.2.1.

\subsection{Solution}

\newpage
\section{4.3.2}

\subsection{Problem}

If the covariant vectors \(\varepsilon_i\) are orthogonal, show that

(a) \(g_{ij}\) is diagonal,

(b) \(g^{ii} = 1/g_{ii}\) (no summation),

(c) \(|\varepsilon^i| = 1/|\varepsilon_i|\).

\subsection{Solution}

\newpage
\section{4.3.3}

\subsection{Problem}

Prove that \((\varepsilon^i \cdot \varepsilon^j)(\varepsilon_j \cdot \varepsilon_k)=\delta^i_k\).

\subsection{Solution}

\newpage
\section{4.3.4}

\subsection{Problem}

Show that \(\Gamma^m_{jk}=\Gamma^m_{kj}\).

\subsection{Solution}

\section{4.3.6}

\subsection{Problem}

Show that the covariant derivative of a covariant vector is given by

\[
    V_{i;j} \equiv \frac{\partial V_i}{\partial q^j} - V_k\Gamma^k_{ij}.
\]

\textit{Hint.} Differentiate

\[
    \varepsilon^i \cdot \varepsilon_j = \delta^i_j.
\]

\subsection{Solution}

\newpage
\bibliographystyle{plain}
\bibliography{references}
\nocite{arfken2013mathematical}
\nocite{El-Deeb_PEU-356_Assignments}

\end{document}