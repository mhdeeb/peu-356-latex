\documentclass[12pt]{article}
\usepackage[svgnames,x11names,table]{xcolor}
\usepackage{hyperref}
\usepackage{graphicx}
\usepackage{parskip}
\usepackage{float}
\usepackage{amsmath}
\usepackage{esint}
\usepackage{amssymb}
\usepackage{enumitem}
\usepackage[thicklines]{cancel}

\hypersetup{
    colorlinks,
    citecolor=blue,
    filecolor=black,
    linkcolor=black,
    urlcolor=RoyalBlue4,
}

\title{PEU 356 Assignment 6}
\author{Mohamed Hussien El-Deeb (201900052)}
\date{\today}

\DeclareMathOperator{\sech}{sech}
\DeclareMathOperator{\csch}{csch}

\begin{document}

\maketitle
\tableofcontents
\hypersetup{linkcolor=RoyalBlue4}

\newpage
\section{5.5.2}

\subsection{Problem}

(a) Given (in \(\mathbb{R}^3\) ) the basis \(\varphi_1=x, \varphi_2=y, \varphi_3=z\), consider the basis transformation \(x \rightarrow z, y \rightarrow y, z \rightarrow-x\). Find the \(3 \times 3\) matrix \(\cup \) for this transformation.

(b) This transformation corresponds to a rotation of the coordinate axes. Identify the rotation and reconcile your transformation matrix with an appropriate matrix \(S(\alpha, \beta, \gamma)\) is of the form,

\[
    \begin{pmatrix}
        \cos \gamma \cos \beta \cos \alpha-\sin \gamma \sin \alpha  & \cos \gamma \cos \beta \sin \alpha+\sin \gamma \cos \alpha  & -\cos \gamma \sin \beta \\
        -\sin \gamma \cos \beta \cos \alpha-\cos \gamma \sin \alpha & -\sin \gamma \cos \beta \sin \alpha+\cos \gamma \cos \alpha & \sin \gamma \sin \beta  \\
        \sin \beta \cos \alpha                                      & \sin \beta \sin \alpha                                      & \cos \beta
    \end{pmatrix}
\]

(c) Form the column vector c representing (in the original basis) \(f=2 x-3 y+z\), find the result of applying \(U\) to \(c\), and show that this is consistent with the basis transformation of part (a).

Note. You do not need to be able to form scalar products to handle this exercise; a knowledge of the linear relationship between the original and transformed functions is sufficient.

\subsection{Solution}

(a)

\[
    \cup = \begin{pmatrix}
        0  & 0 & 1 \\
        0  & 1 & 0 \\
        -1 & 0 & 0
    \end{pmatrix}
\]

(b)

\[
    \cup \vec{v} = \vec{v}
\]

\[
    \begin{pmatrix}
        0  & 0 & 1 \\
        0  & 1 & 0 \\
        -1 & 0 & 0
    \end{pmatrix}
    \begin{pmatrix}
        x \\
        y \\
        z
    \end{pmatrix}
    =
    \begin{pmatrix}
        x \\
        y \\
        z
    \end{pmatrix}
\]

\[
    z = x = 0, \quad y = y
\]

\[
    \hat{n} = \begin{pmatrix}
        0 \\
        1 \\
        0
    \end{pmatrix}
\]

\[
    \mathbf{r}^{\prime}=\mathbf{r} \cos \boldsymbol{\Phi}+\mathbf{r} \times \hat{\mathbf{n}} \sin \boldsymbol{\Phi}+\hat{\mathbf{n}}(\hat{\mathbf{n}} \cdot \mathbf{r})(1-\cos \boldsymbol{\Phi})
\].


\[
    \mathbf{r}^{\prime}=(r_x\hat{\imath} + r_y\hat{\jmath} + r_z\hat{k}) \cos \boldsymbol{\Phi}+(r_x\hat{k}-r_z\hat{\imath}) \sin \boldsymbol{\Phi}+r_y(1-\cos \boldsymbol{\Phi})\hat{\jmath}
\]

\[
    \mathbf{r}^{\prime}=
    \left\langle r_x\cos \boldsymbol{\Phi} - r_z\sin \boldsymbol{\Phi}
    , r_y
    , r_x\sin \boldsymbol{\Phi} + r_z\cos \boldsymbol{\Phi}\right\rangle
\]

\[
    T = \begin{pmatrix}
        \cos \boldsymbol{\Phi} & 0 & - \sin \boldsymbol{\Phi} \\
        0                      & 1 & 0                        \\
        \sin \boldsymbol{\Phi} & 0 & \cos \boldsymbol{\Phi}
    \end{pmatrix}
\]

\[
    \boldsymbol{\Phi} = \frac{3\pi}{2}
\]

(C)

\[
    \begin{pmatrix}
        0  & 0 & 1 \\
        0  & 1 & 0 \\
        -1 & 0 & 0
    \end{pmatrix}
    \begin{pmatrix}
        2  \\
        -3 \\
        1
    \end{pmatrix}
    =
    \begin{pmatrix}
        1  \\
        -3 \\
        -2
    \end{pmatrix}
\]

\[
    f = 2 \varphi_1 - 3 \varphi_2 + 1 \varphi_3 \rightarrow f' = 1 \varphi_1^{\prime} - 3 \varphi_2^{\prime} - 2 \varphi_3^{\prime}
\]

It is consistent with the basis transformation of part (a).

\newpage
\section{5.5.4}

\subsection{Problem}

The unitary transformation \(U\) that converts an orthonormal basis \(\left\{\varphi_i\right\}\) into the basis \(\left\{\varphi_i^{\prime}\right\}\) and the unitary transformation \(V\) that converts the basis \(\left\{\varphi_i^{\prime}\right\}\) into the basis \(\left\{\chi_i\right\}\) have matrix representations

\[
    \mathrm{U}=\left(\begin{array}{ccc}
            i \sin \theta & \cos \theta   & 0 \\
            -\cos \theta  & i \sin \theta & 0 \\
            0             & 0             & 1
        \end{array}\right), \quad \mathrm{V}=\left(\begin{array}{ccc}
            1 & 0           & 0              \\
            0 & \cos \theta & i \sin \theta  \\
            0 & \cos \theta & -i \sin \theta
        \end{array}\right) .
\]

Given the function \(f(x)=3 \varphi_1(x)-\varphi_2(x)-2 \varphi_3(x)\),

(a) By applying \(\mathrm{U}\), form the vector representing \(f(x)\) in the \(\left\{\varphi_i^{\prime}\right\}\) basis and then by applying \(\mathrm{V}\) form the vector representing \(f(x)\) in the \(\left\{\chi_i\right\}\) basis. Use this result to write \(f(x)\) as a linear combination of the \(\chi_i\).

(b) Form the matrix products UV and VU and then apply each to the vector representing \(f(x)\) in the \(\left\{\varphi_i\right\}\) basis. Verify that the results of these applications differ and that only one of them gives the result corresponding to part (a).

\subsection{Solution}

\[
    \vec{f} = \begin{pmatrix}
        3  \\
        -1 \\
        -2
    \end{pmatrix}
\]

(a)

\[
    \begin{pmatrix}
        i \sin \theta & \cos \theta   & 0 \\
        -\cos \theta  & i \sin \theta & 0 \\
        0             & 0             & 1
    \end{pmatrix}
    \begin{pmatrix}
        3  \\
        -1 \\
        -2
    \end{pmatrix}
    =
    \begin{pmatrix}
        3 i \sin \theta - \cos \theta  \\
        -3 \cos \theta - i \sin \theta \\
        -2
    \end{pmatrix}
\]

\[
    \begin{pmatrix}
        1 & 0           & 0              \\
        0 & \cos \theta & i \sin \theta  \\
        0 & \cos \theta & -i \sin \theta
    \end{pmatrix}
    \begin{pmatrix}
        3 i \sin \theta - \cos \theta  \\
        -3 \cos \theta - i \sin \theta \\
        -2
    \end{pmatrix} =
\]

\[
    \begin{pmatrix}
        3 i \sin \theta - \cos \theta                                \\
        \cos \theta (-3 \cos \theta - i \sin \theta) - 2i\sin \theta \\
        \cos \theta (-3 \cos \theta - i \sin \theta) + 2i\sin \theta
    \end{pmatrix}
\]

\[
    = \begin{pmatrix}
        3 i \sin \theta - \cos \theta                                \\
        -3 \cos^2 \theta - i \sin \theta \cos \theta - 2i\sin \theta \\
        -3 \cos^2 \theta - i \sin \theta \cos \theta + 2i\sin \theta
    \end{pmatrix}
\]

\[
    f(x) = \left(3 i \sin \theta - \cos \theta\right) \chi_1(x) +
\]

\[
    \left(\cos \theta (-3 \cos \theta - i \sin \theta) - 2i\sin \theta\right) \chi_2(x) +
\]

\[
    \left(\cos \theta (-3 \cos \theta - i \sin \theta) + 2i\sin \theta\right) \chi_3(x)
\]

(b)

UV was already implicitly calculated in part (a).

\[
    VU = \begin{pmatrix}
        1 & 0           & 0              \\
        0 & \cos \theta & i \sin \theta  \\
        0 & \cos \theta & -i \sin \theta
    \end{pmatrix}
    \begin{pmatrix}
        i \sin \theta & \cos \theta   & 0 \\
        -\cos \theta  & i \sin \theta & 0 \\
        0             & 0             & 1
    \end{pmatrix}
\]

\[
    = \begin{pmatrix}
        i \sin \theta  & \cos \theta               & 0              \\
        -\cos^2 \theta & i \sin \theta \cos \theta & i \sin \theta  \\
        -\cos^2 \theta & i \sin \theta \cos \theta & -i \sin \theta
    \end{pmatrix}
\]

\[
    \begin{pmatrix}
        i \sin \theta  & \cos \theta               & 0              \\
        -\cos^2 \theta & i \sin \theta \cos \theta & i \sin \theta  \\
        -\cos^2 \theta & i \sin \theta \cos \theta & -i \sin \theta
    \end{pmatrix}
    \begin{pmatrix}
        3  \\
        -1 \\
        -2
    \end{pmatrix}
\]

\[
    = \begin{pmatrix}
        3 i \sin \theta - \cos \theta                           \\
        -3 \cos^2 - i \sin \theta \cos \theta + 2 i \sin \theta \\
        -3 \cos^2 - i \sin \theta \cos \theta - 2 i \sin \theta
    \end{pmatrix}
\]

There is a slight difference between the two results, and only the result of UV corresponds to part (a).

\[
    {\left(UV\vec{f}\right)} _2 = {\left(VU\vec{f}\right)} _3, \quad {\left(UV\vec{f}\right)} _3 = {\left(VU\vec{f}\right)} _2
\]


\newpage
\section{5.6.1}

\subsection{Problem}

(a) Using the two spin functions \(\varphi_1=\alpha\) and \(\varphi_2=\beta\) as an orthonormal basis (so \(\langle\alpha \mid \alpha\rangle=\langle\beta \mid \beta\rangle=1,\langle\alpha \mid \beta\rangle=0)\), and the relations

\[
    S_x \alpha=\frac{1}{2} \beta, \quad S_x \beta=\frac{1}{2} \alpha, \quad S_y \alpha=\frac{1}{2} i \beta, \quad S_y \beta=-\frac{1}{2} i \alpha, \quad S_z \alpha=\frac{1}{2} \alpha, \quad S_z \beta=-\frac{1}{2} \beta,
\]

construct the \(2 \times 2\) matrices of \(S_x, S_y\), and \(S_z\).

(b) Taking now the basis \(\varphi_1^{\prime}=C(\alpha+\beta), \varphi_2^{\prime}=C(\alpha-\beta)\):

(i) Verify that \(\varphi_1^{\prime}\) and \(\varphi_2^{\prime}\) are orthogonal,

(ii) Assign \(C\) a value that makes \(\varphi_1^{\prime}\) and \(\varphi_2^{\prime}\) normalized,

(iii) Find the unitary matrix for the transformation \(\left\{\varphi_i\right\} \rightarrow\left\{\varphi_i^{\prime}\right\}\).

(c) Find the matrices of \(S_x, S_y\), and \(S_z\) in the \(\left\{\varphi_i^{\prime}\right\}\) basis.

\subsection{Solution}

(a)

\[
    S_x | \alpha \rangle = \frac{1}{2} | \beta \rangle
    \rightarrow  S_x | \alpha \rangle \langle \alpha |
    = \frac{1}{2} | \beta \rangle \langle \alpha | \longrightarrow (1)
\]

\[
    S_x | \beta \rangle = \frac{1}{2} | \alpha \rangle
    \rightarrow  S_x | \beta \rangle \langle \beta |
    = \frac{1}{2} | \alpha \rangle \langle \beta | \longrightarrow (2)
\]

\[
    (1) + (2) \longrightarrow S_x (| \alpha \rangle \langle \alpha | + | \beta \rangle \langle \beta |)
    = \frac{1}{2} (| \beta \rangle \langle \alpha | + | \alpha \rangle \langle \beta |)
\]

\[
    \because \sum_i | \varphi_i \rangle \langle \varphi_i | = I
\]

\[
    S_x = \frac{1}{2} (| \beta \rangle \langle \alpha | + | \alpha \rangle \langle \beta |)
\]

\[
    S_x =
    \begin{pmatrix}
        \alpha_1\beta_1                             & \frac{\alpha_1\beta_2 + \alpha_2\beta_1}{2} \\
        \frac{\alpha_1\beta_2 + \alpha_2\beta_1}{2} & \alpha_2\beta_2
    \end{pmatrix}
\]

\[
    S_y | \alpha \rangle = \frac{i}{2} | \beta \rangle
    \rightarrow  S_y | \alpha \rangle \langle \alpha |
    = \frac{i}{2} | \beta \rangle \langle \alpha | \longrightarrow (1)
\]

\[
    S_y | \beta \rangle = - \frac{i}{2} | \alpha \rangle
    \rightarrow  S_y | \beta \rangle \langle \beta |
    = - \frac{i}{2} | \alpha \rangle \langle \beta | \longrightarrow (2)
\]

\[
    (1) + (2) \longrightarrow S_y (| \alpha \rangle \langle \alpha | + | \beta \rangle \langle \beta |)
    = \frac{i}{2} (| \beta \rangle \langle \alpha | - | \alpha \rangle \langle \beta |)
\]

\[
    S_y = \frac{i}{2} (| \beta \rangle \langle \alpha | - | \alpha \rangle \langle \beta |)
\]

\[
    S_y =
    \begin{pmatrix}
        0                                              & \frac{i(\alpha_1\beta_2 - \alpha_2\beta_1)}{2} \\
        \frac{i(\alpha_2\beta_1 - \alpha_1\beta_2)}{2} & 0
    \end{pmatrix}
\]

\[
    S_z | \alpha \rangle = \frac{1}{2} | \alpha \rangle
    \rightarrow  S_z | \alpha \rangle \langle \alpha |
    = \frac{1}{2} | \alpha \rangle \langle \alpha | \longrightarrow (1)
\]

\[
    S_z | \beta \rangle = - \frac{1}{2} | \beta \rangle
    \rightarrow  S_z | \beta \rangle \langle \beta |
    = - \frac{1}{2} | \beta \rangle \langle \beta | \longrightarrow (2)
\]

\[
    (1) + (2) \longrightarrow S_z (| \alpha \rangle \langle \alpha | + | \beta \rangle \langle \beta |)
    = \frac{1}{2} (| \alpha \rangle \langle \alpha | - | \beta \rangle \langle \beta |)
\]

\[
    S_z = \frac{1}{2} (| \alpha \rangle \langle \alpha | - | \beta \rangle \langle \beta |)
\]

\[
    S_z =
    \begin{pmatrix}
        \frac{\alpha_1^2 - \beta_1^2}{2}          & \frac{\alpha_1\alpha_2-\beta_1\beta_2}{2} \\
        \frac{\alpha_1\alpha_2-\beta_1\beta_2}{2} & \frac{\alpha_2^2 - \beta_2^2}{2}
    \end{pmatrix}
\]

(b)

(i)

If we assume \(C = \frac{1}{\sqrt{2}}U\) where U is a unitary matrix.

\[
    C^\dagger C = {\left(\frac{1}{\sqrt{2}}U\right) }^\dagger\frac{1}{\sqrt{2}}U
    = U^\dagger \frac{1}{\sqrt{2}} \frac{1}{\sqrt{2}} U
    = \frac{1}{2} U^\dagger U
    = \frac{1}{2} I
\]

\[
    \langle \varphi_1^{\prime} | \varphi_2^{\prime} \rangle
    = \langle C^\dagger C(\alpha + \beta) | \alpha - \beta \rangle
    = \frac{1}{2} \langle \alpha + \beta | \alpha - \beta \rangle
\]

\[
    = \frac{1}{2} \left(
    \langle \alpha | \alpha \rangle
    - \langle \alpha | \beta \rangle
    + \langle \beta | \alpha \rangle
    - \langle \beta | \beta \rangle
    \right) = 0
\]

Note: if we want C to be constant U would just be the identity matrix.

(ii)

\[
    C = \frac{1}{\sqrt{2}}U
\]

Where U is a unitary matrix.

\[
    \langle \varphi_1^{\prime} | \varphi_1^{\prime} \rangle
    = \langle C(\alpha + \beta) | C(\alpha + \beta) \rangle
    = \langle C^\dagger C(\alpha + \beta) | (\alpha + \beta) \rangle
\]

\[
    = \frac{1}{2}\langle (\alpha + \beta) | (\alpha + \beta) \rangle
    = \frac{1}{2}(\langle \alpha | \alpha \rangle + \langle \beta | \beta \rangle + \langle \alpha | \beta \rangle + \langle \beta | \alpha \rangle)
    = 1
\]

\[
    \langle \varphi_2^{\prime} | \varphi_2^{\prime} \rangle
    = \langle C(\alpha - \beta) | C(\alpha - \beta) \rangle
    = \langle C^\dagger C(\alpha - \beta) | (\alpha - \beta) \rangle
\]

\[
    = \frac{1}{2}\langle (\alpha - \beta) | (\alpha - \beta) \rangle
    = \frac{1}{2}(\langle \alpha | \alpha \rangle + \langle \beta | \beta \rangle - \langle \alpha | \beta \rangle - \langle \beta | \alpha \rangle)
    = 1
\]

(iii)

\[
    T =
    \begin{pmatrix}
        \frac{1}{\sqrt{2}} & \frac{1}{\sqrt{2}}  \\
        \frac{1}{\sqrt{2}} & -\frac{1}{\sqrt{2}}
    \end{pmatrix}
\]

Note: there is a mistake in the question, T is not a unitary matrix.
If we want T to be unitary, then \(\varphi_2 = -C(\alpha - \beta)\),
there should be a negative sign that was not included.
This will not have an effect on the rest of the answers.

If we follow the fix T would be a unitary matrix.

\[
    T =
    \begin{pmatrix}
        \frac{1}{\sqrt{2}}  & \frac{1}{\sqrt{2}} \\
        -\frac{1}{\sqrt{2}} & \frac{1}{\sqrt{2}}
    \end{pmatrix}
\]

(c)

We simply need to substitute \(\alpha_i \rightarrow \frac{1}{\sqrt{2}}(\alpha_i + \beta_i)\) and \(\beta_i \rightarrow \frac{1}{\sqrt{2}}(\alpha_i - \beta_i)\) into the matrices of \(S_x, S_y\), and \(S_z\).

\[
    S_x =
    \begin{pmatrix}
        \alpha_1\beta_1                             & \frac{\alpha_1\beta_2 + \alpha_2\beta_1}{2} \\
        \frac{\alpha_1\beta_2 + \alpha_2\beta_1}{2} & \alpha_2\beta_2
    \end{pmatrix}
\]

\[
    S_y =
    \begin{pmatrix}
        0                                               & \frac{i(\alpha_1\beta_2 - \alpha_2\beta_1)}{2} \\
        -\frac{i(\alpha_1\beta_2 - \alpha_2\beta_1)}{2} & 0
    \end{pmatrix}
\]

\[
    S_z =
    \begin{pmatrix}
        \frac{\alpha_1^2 - \beta_1^2}{2}          & \frac{\alpha_1\alpha_2-\beta_1\beta_2}{2} \\
        \frac{\alpha_1\alpha_2-\beta_1\beta_2}{2} & \frac{\alpha_2^2 - \beta_2^2}{2}
    \end{pmatrix}
\]

\[
    {S_x}' =
    \begin{pmatrix}
        \frac{1}{\sqrt{2}}(\alpha_1 + \beta_1)\frac{1}{\sqrt{2}}(\alpha_1 - \beta_1)                                                                                          & \frac{\frac{1}{\sqrt{2}}(\alpha_1 + \beta_1)\frac{1}{\sqrt{2}}(\alpha_2 - \beta_2) + \frac{1}{\sqrt{2}}(\alpha_2 + \beta_2)\frac{1}{\sqrt{2}}(\alpha_1 - \beta_1)}{2} \\
        \frac{\frac{1}{\sqrt{2}}(\alpha_1 + \beta_1)\frac{1}{\sqrt{2}}(\alpha_2 - \beta_2) + \frac{1}{\sqrt{2}}(\alpha_2 + \beta_2)\frac{1}{\sqrt{2}}(\alpha_1 - \beta_1)}{2} & \frac{1}{\sqrt{2}}(\alpha_2 + \beta_2)\frac{1}{\sqrt{2}}(\alpha_2 - \beta_2)
    \end{pmatrix}
\]

\[
    {S_x}' =
    \begin{pmatrix}
        \frac{1}{2}(\alpha_1^2 - \beta_1^2)         & \frac{\alpha_1\alpha_2 - \beta_1\beta_2}{2} \\
        \frac{\alpha_1\alpha_2 - \beta_1\beta_2}{2} & \frac{1}{2}(\alpha_2^2 - \beta_2^2)
    \end{pmatrix}
\]

\[
    \alpha_1\beta_2 - \alpha_2\beta_1 \rightarrow
    \frac{1}{2}\left((\alpha_1 + \beta_1)(\alpha_2 - \beta_2) - (\alpha_2 + \beta_2)(\alpha_1 - \beta_1)\right)
\]

\[
    = \alpha_2\beta_1 - \alpha_1\beta_2
\]

\[
    {S_y}' =
    \begin{pmatrix}
        0                                              & -\frac{i(\alpha_1\beta_2 - \alpha_2\beta_1)}{2} \\
        \frac{i(\alpha_1\beta_2 - \alpha_2\beta_1)}{2} & 0
    \end{pmatrix}
\]

\[
    {S_z}' =
    \begin{pmatrix}
        \alpha_1\beta_1                             & \frac{\alpha_1\beta_2 + \alpha_2\beta_1}{2} \\
        \frac{\alpha_1\beta_2 + \alpha_2\beta_1}{2} & \alpha_2\beta_2
    \end{pmatrix}
\]

\newpage
\section{5.6.2}

\subsection{Problem}

For the basis \(\varphi_1=C x e^{-r^2}, \varphi_2=C y e^{-r^2}, \varphi_3=C z e^{-r^2}\), where \(r^2=x^2+y^2+z^2\), with the scalar product defined as an unweighted integral over \(\mathbb{R}^3\) and with \(C\) chosen to make the \(\varphi_i\) normalized:

(a) Find the \(3 \times 3\) matrix of \(L_x=-i\left(y \frac{\partial}{\partial z}-z \frac{\partial}{\partial y}\right)\);

(b) Using the transformation matrix \(\mathrm{U}=\left(\begin{array}{ccc}1 & 0 & 0 \\ 0 & 1 / \sqrt{2} & -i / \sqrt{2} \\ 0 & 1 / \sqrt{2} & i / \sqrt{2}\end{array}\right)\), find the transformed matrix of \(L_x\);

(c) Find the new basis functions \(\varphi_i^{\prime}\) defined by the transformation \(\mathrm{U}\), and write explicitly (in terms of \(x, y\), and \(z\) ) the functional forms of \(L_x \varphi_i^{\prime}, i=1,2,3\).

Hint. Use \(\int e^{-r^2} d^3 r=\pi^{3 / 2}, \int x^2 e^{-r^2} d^3 r=\frac{1}{2} \pi^{3 / 2} ;\) the integrals are over \(\mathbb{R}^3\).

\subsection{Solution}

(a)

\[
    A_{n m}=\left\langle\phi_n|\hat{A}| \phi_m\right\rangle
\]

\[
    L_x = -i\left(y \frac{\partial}{\partial z}-z \frac{\partial}{\partial y}\right)
\]

\[
    \left\langle\phi_n|L_x| \phi_m\right\rangle
    = -i\left\langle\phi_n\left|y \frac{\partial}{\partial z}-z \frac{\partial}{\partial y}\right| \phi_m\right\rangle
\]

\[
    \left\langle\phi_n|\phi_m\right\rangle = \int \phi_n^* \phi_m d^3 r
\]

\[
    \left\langle\phi_n|L_x| \phi_m\right\rangle
    = -i\int \phi_n^* \left(y \frac{\partial}{\partial z}-z \frac{\partial}{\partial y}\right) \phi_m d^3 r
\]

\[
    = -i{\left\lvert C\right\rvert}^2 \int x_n e^{-r^2} \left(x_2 \frac{\partial}{\partial x_3}-x_3 \frac{\partial}{\partial x_2}\right) x_m e^{-r^2} d^3 r
\]

\[
    = -i{\left\lvert C\right\rvert}^2 \int x_n e^{-r^2} \left(x_2 \frac{\partial \left(x_m e^{-r^2}\right) }{\partial x_3}-x_3 \frac{\partial \left(x_m e^{-r^2}\right) }{\partial x_2}\right) d^3 r
\]

\[
    = -i{\left\lvert C\right\rvert}^2 \int x_n e^{-r^2} \left(x_2 \left(\delta^m_3 e^{-r^2} + x_m \frac{\partial e^{-r^2}}{\partial x_3}\right) - x_3 \left(\delta^m_2 e^{-r^2} + x_m \frac{\partial e^{-r^2}}{\partial x_2}\right)\right) d^3 r
\]

\[
    \frac{\partial e^{-r^2}}{\partial x_i} = -2 r \frac{\partial r}{\partial x_i} e^{-r^2}
    = -2 x_i e^{-r^2}
\]

\[
    = -i{\left\lvert C\right\rvert}^2 \int x_n e^{-2r^2} \left(x_2 \left(\delta^m_3 - 2 x_m x_3\right) - x_3 \left(\delta^m_2 - 2 x_m x_2\right)\right) d^3 r
\]

\[
    L_{nm} = i{\left\lvert C\right\rvert}^2 \int e^{-2r^2} x_n \left(x_3 \delta^m_2 - x_2 \delta^m_3\right)  d^3 r
\]

\[
    L_{nm} = i{\left\lvert C\right\rvert}^2
    \int_{-\infty }^{\infty}
    \int_{-\infty}^{\infty}
    \int_{-\infty}^{\infty}
    e^{-2{x_1}^2}e^{-2{x_2}^2}e^{-2{x_3}^2}
    x_n \left(x_3 \delta^m_2 - x_2 \delta^m_3\right)  d{x_1} d{x_2} d{x_3}
\]

From this we can see that only the terms with \(n = 2, m = 3\) and \(n = 3, m = 2\) will survive.
Because they are the product of two even functions, the rest are have odd symmetry and will integrate to zero.

\[
    L_{23} = -i{\left\lvert C\right\rvert}^2
    \int_{-\infty }^{\infty}
    \int_{-\infty}^{\infty}
    \int_{-\infty}^{\infty}
    e^{-2{x_1}^2}e^{-2{x_2}^2}e^{-2{x_3}^2}
    {x_2}^2  d{x_1} d{x_2} d{x_3}
\]

\[
    = -i{\left\lvert C\right\rvert}^2
    \int_{-\infty }^{\infty}e^{-2{x_1}^2}d{x_1}
    \int_{-\infty}^{\infty}{x_2}^2e^{-2{x_2}^2}d{x_2}
    \int_{-\infty}^{\infty}e^{-2{x_3}^2}d{x_3}
\]

\[
    = -i{\left\lvert C\right\rvert}^2
    \left(\frac{\pi^{\frac{3}{2}}}{8\sqrt{2}}\right)
\]

\[
    {\left\lvert C\right\rvert}^2 = \frac{8\sqrt{2}}{\pi^{\frac{3}{2}}}
\]

\[
    L_{23} = -i
\]

\[
    L_{32} = i \frac{8\sqrt{2}}{\pi^{\frac{3}{2}}}
    \int_{-\infty }^{\infty}
    \int_{-\infty}^{\infty}
    \int_{-\infty}^{\infty}
    e^{-2{x_1}^2}e^{-2{x_2}^2}e^{-2{x_3}^2}
    {x_3}^2  d{x_1} d{x_2} d{x_3}
\]

\[
    = i \frac{8\sqrt{2}}{\pi^{\frac{3}{2}}}
    \int_{-\infty }^{\infty}e^{-2{x_1}^2}d{x_1}
    \int_{-\infty}^{\infty}e^{-2{x_2}^2}d{x_2}
    \int_{-\infty}^{\infty}{x_3}^2e^{-2{x_3}^2}d{x_3}
\]

\[
    = i \frac{8\sqrt{2}}{\pi^{\frac{3}{2}}}
    \left(\frac{\pi^{\frac{3}{2}}}{8\sqrt{2}}\right)
    = i
\]

\[
    L_x =
    \begin{pmatrix}
        0 & 0 & 0  \\
        0 & 0 & -i \\
        0 & i & 0
    \end{pmatrix}
\]

(b)



\newpage
\section{5.7.1}

\subsection{Problem}

Using the formal properties of unitary transformations, show that the commutator \([x, p]=i\) is invariant under unitary transformation of the matrices representing \(x\) and \(p\).

\subsection{Solution}

\[
    [x, p] = i
\]

\[
    U x U^\dagger = x'
\]

\[
    U p U^\dagger = p'
\]

\[
    [x', p'] =U x U^\dagger U p U^\dagger - U p U^\dagger U x U^\dagger
\]

\[
    = U x p U^\dagger - U p x U^\dagger = U [x, p] U^\dagger = U i U^\dagger = i
\]

\newpage
\section{5.7.2}

\subsection{Problem}

The Pauli matrices
\[
    \sigma_1=\left(\begin{array}{ll}
            0 & 1 \\
            1 & 0
        \end{array}\right), \quad \sigma_2=\left(\begin{array}{cc}
            0 & -i \\
            i & 0
        \end{array}\right), \quad \sigma_3=\left(\begin{array}{rr}
            1 & 0  \\
            0 & -1
        \end{array}\right),
\]
have commutator \(\left[\sigma_1, \sigma_2\right]=2 i \sigma_3\). Show that this relationship continues to be valid if these matrices are transformed by
\[
    \mathbf{U}=\left(\begin{array}{rr}
            \cos \theta  & \sin \theta \\
            -\sin \theta & \cos \theta
        \end{array}\right) .
\]

\subsection{Solution}

\[
    \mathbf{U}\sigma_1\mathbf{U^\dagger} =
    \begin{pmatrix}
        \cos \theta  & \sin \theta \\
        -\sin \theta & \cos \theta
    \end{pmatrix}
    \begin{pmatrix}
        0 & 1 \\
        1 & 0
    \end{pmatrix}
    \begin{pmatrix}
        \cos \theta & -\sin \theta \\
        \sin \theta & \cos \theta
    \end{pmatrix}
\]

\[
    =
    \begin{pmatrix}
        \sin \theta & \cos \theta  \\
        \cos \theta & -\sin \theta
    \end{pmatrix}
    \begin{pmatrix}
        \cos \theta & -\sin \theta \\
        \sin \theta & \cos \theta
    \end{pmatrix}
\]

\[
    =
    \begin{pmatrix}
        \sin 2\theta  & \cos 2 \theta \\
        \cos 2 \theta & -\sin 2\theta
    \end{pmatrix}
\]

\[
    \mathbf{U}\sigma_2\mathbf{U^\dagger} =
    \begin{pmatrix}
        \cos \theta  & \sin \theta \\
        -\sin \theta & \cos \theta
    \end{pmatrix}
    \begin{pmatrix}
        0 & -i \\
        i & 0
    \end{pmatrix}
    \begin{pmatrix}
        \cos \theta & -\sin \theta \\
        \sin \theta & \cos \theta
    \end{pmatrix}
\]

\[
    =
    \begin{pmatrix}
        i\sin \theta & -i\cos \theta \\
        i\cos \theta & i\sin \theta
    \end{pmatrix}
    \begin{pmatrix}
        \cos \theta & -\sin \theta \\
        \sin \theta & \cos \theta
    \end{pmatrix}
\]

\[
    =
    \begin{pmatrix}
        0 & -i \\
        i & 0
    \end{pmatrix}
\]

\[
    \mathbf{U}\sigma_3\mathbf{U^\dagger} =
    \begin{pmatrix}
        \cos \theta  & \sin \theta \\
        -\sin \theta & \cos \theta
    \end{pmatrix}
    \begin{pmatrix}
        1 & 0  \\
        0 & -1
    \end{pmatrix}
    \begin{pmatrix}
        \cos \theta & -\sin \theta \\
        \sin \theta & \cos \theta
    \end{pmatrix}
\]

\[
    =
    \begin{pmatrix}
        \cos \theta  & -\sin \theta \\
        -\sin \theta & -\cos \theta
    \end{pmatrix}
    \begin{pmatrix}
        \cos \theta & -\sin \theta \\
        \sin \theta & \cos \theta
    \end{pmatrix}
\]

\[
    =
    \begin{pmatrix}
        \cos 2 \theta  & -\sin 2 \theta \\
        -\sin 2 \theta & -\cos 2 \theta
    \end{pmatrix}
\]

\[
    \left[\mathbf{U}\sigma_1\mathbf{U^\dagger}, \mathbf{U}\sigma_2\mathbf{U^\dagger}\right] =
    \begin{pmatrix}
        \sin 2\theta  & \cos 2 \theta \\
        \cos 2 \theta & -\sin 2\theta
    \end{pmatrix}
    \begin{pmatrix}
        0 & -i \\
        i & 0
    \end{pmatrix}
    -
    \begin{pmatrix}
        0 & -i \\
        i & 0
    \end{pmatrix}
    \begin{pmatrix}
        \sin 2\theta  & \cos 2 \theta \\
        \cos 2 \theta & -\sin 2\theta
    \end{pmatrix}
\]

\[
    =
    \begin{pmatrix}
        i\cos 2\theta  & -i\sin 2\theta \\
        -i\sin 2\theta & -i\cos 2\theta
    \end{pmatrix}
    -
    \begin{pmatrix}
        -i\cos 2\theta & i\sin 2\theta \\
        i\sin 2\theta  & i\cos 2\theta
    \end{pmatrix}
\]

\[
    =
    \begin{pmatrix}
        2i\cos 2\theta  & -2i\sin 2\theta \\
        -2i\sin 2\theta & -2i\cos 2\theta
    \end{pmatrix}
\]

\[
    = 2i\mathbf{U}\sigma_3\mathbf{U^\dagger}
\]

\newpage
\bibliographystyle{plain}
\bibliography{references}
\nocite{El-Deeb_PEU-356_Assignments}

\end{document}