\documentclass[12pt]{article}
\usepackage[svgnames,x11names,table]{xcolor}
\usepackage{hyperref}
\usepackage{graphicx}
\usepackage{parskip}
\usepackage{float}
\usepackage{amsmath}
\usepackage{esint}
\usepackage{amssymb}
\usepackage{enumitem}
\usepackage[thicklines]{cancel}

\hypersetup{
    colorlinks,
    citecolor=blue,
    filecolor=black,
    linkcolor=black,
    urlcolor=RoyalBlue4,
}

\title{PEU 356 Assignment 6}
\author{Mohamed Hussien El-Deeb (201900052)}
\date{\today}

\DeclareMathOperator{\sech}{sech}
\DeclareMathOperator{\csch}{csch}

\begin{document}

\maketitle
\tableofcontents
\hypersetup{linkcolor=RoyalBlue4}

\newpage
\section{Question 1}

\subsection{Problem}

(a) Given (in \(\mathbb{R}^3\) ) the basis \(\varphi_1=x, \varphi_2=y, \varphi_3=z\), consider the basis transformation \(x \rightarrow z, y \rightarrow y, z \rightarrow-x\). Find the \(3 \times 3\) matrix \(\cup \) for this transformation.

(b) This transformation corresponds to a rotation of the coordinate axes. Identify the rotation and reconcile your transformation matrix with an appropriate matrix \(S(\alpha, \beta, \gamma)\) is of the form,

\[
    \begin{pmatrix}
        \cos \gamma \cos \beta \cos \alpha-\sin \gamma \sin \alpha  & \cos \gamma \cos \beta \sin \alpha+\sin \gamma \cos \alpha  & -\cos \gamma \sin \beta \\
        -\sin \gamma \cos \beta \cos \alpha-\cos \gamma \sin \alpha & -\sin \gamma \cos \beta \sin \alpha+\cos \gamma \cos \alpha & \sin \gamma \sin \beta  \\
        \sin \beta \cos \alpha                                      & \sin \beta \sin \alpha                                      & \cos \beta
    \end{pmatrix}
\]

(c) Form the column vector c representing (in the original basis) \(f=2 x-3 y+z\), find the result of applying \(U\) to \(c\), and show that this is consistent with the basis transformation of part (a).

Note. You do not need to be able to form scalar products to handle this exercise; a knowledge of the linear relationship between the original and transformed functions is sufficient.

\subsection{Solution}



\newpage
\section{Question 2}

\subsection{Problem}

The unitary transformation \(U\) that converts an orthonormal basis \(\left\{\varphi_i\right\}\) into the basis \(\left\{\varphi_i^{\prime}\right\}\) and the unitary transformation \(V\) that converts the basis \(\left\{\varphi_i^{\prime}\right\}\) into the basis \(\left\{\chi_i\right\}\) have matrix representations

\[
    \mathrm{U}=\left(\begin{array}{ccc}
            i \sin \theta & \cos \theta   & 0 \\
            -\cos \theta  & i \sin \theta & 0 \\
            0             & 0             & 1
        \end{array}\right), \quad \mathrm{V}=\left(\begin{array}{ccc}
            1 & 0           & 0              \\
            0 & \cos \theta & i \sin \theta  \\
            0 & \cos \theta & -i \sin \theta
        \end{array}\right) .
\]

Given the function \(f(x)=3 \varphi_1(x)-\varphi_2(x)-2 \varphi_3(x)\),

(a) By applying \(\mathrm{U}\), form the vector representing \(f(x)\) in the \(\left\{\varphi_i^{\prime}\right\}\) basis and then by applying \(\mathrm{V}\) form the vector representing \(f(x)\) in the \(\left\{\chi_i\right\}\) basis. Use this result to write \(f(x)\) as a linear combination of the \(\chi_i\).

(b) Form the matrix products UV and VU and then apply each to the vector representing \(f(x)\) in the \(\left\{\varphi_i\right\}\) basis. Verify that the results of these applications differ and that only one of them gives the result corresponding to part (a).

\subsection{Solution}



\newpage
\section{Question 3}

\subsection{Problem}

(a) Using the two spin functions \(\varphi_1=\alpha\) and \(\varphi_2=\beta\) as an orthonormal basis (so \(\langle\alpha \mid \alpha\rangle=\langle\beta \mid \beta\rangle=1,\langle\alpha \mid \beta\rangle=0)\), and the relations

\[
    S_x \alpha=\frac{1}{2} \beta, \quad S_x \beta=\frac{1}{2} \alpha, \quad S_y \alpha=\frac{1}{2} i \beta, \quad S_y \beta=-\frac{1}{2} i \alpha, \quad S_z \alpha=\frac{1}{2} \alpha, \quad S_z \beta=-\frac{1}{2} \beta,
\]

construct the \(2 \times 2\) matrices of \(S_x, S_y\), and \(S_z\).
(b) Taking now the basis \(\varphi_1^{\prime}=C(\alpha+\beta), \varphi_2^{\prime}=C(\alpha-\beta)\) :

(i) Verify that \(\varphi_1^{\prime}\) and \(\varphi_2^{\prime}\) are orthogonal,

(ii) Assign \(C\) a value that makes \(\varphi_1^{\prime}\) and \(\varphi_2^{\prime}\) normalized,

(iii) Find the unitary matrix for the transformation \(\left\{\varphi_i\right\} \rightarrow\left\{\varphi_i^{\prime}\right\}\).

(c) Find the matrices of \(S_x, S_y\), and \(S_z\) in the \(\left\{\varphi_i^{\prime}\right\}\) basis.

\subsection{Solution}


\newpage
\section{Question 4}

\subsection{Problem}

For the basis \(\varphi_1=C x e^{-r^2}, \varphi_2=C y e^{-r^2}, \varphi_3=C z e^{-r^2}\), where \(r^2=x^2+y^2+z^2\), with the scalar product defined as an unweighted integral over \(\mathbb{R}^3\) and with \(C\) chosen to make the \(\varphi_i\) normalized:

(a) Find the \(3 \times 3\) matrix of \(L_x=-i\left(y \frac{\partial}{\partial z}-z \frac{\partial}{\partial y}\right)\);

(b) Using the transformation matrix \(\mathrm{U}=\left(\begin{array}{ccc}1 & 0 & 0 \\ 0 & 1 / \sqrt{2} & -i / \sqrt{2} \\ 0 & 1 / \sqrt{2} & i / \sqrt{2}\end{array}\right)\), find the transformed matrix of \(L_x\);

(c) Find the new basis functions \(\varphi_i^{\prime}\) defined by the transformation \(\mathrm{U}\), and write explicitly (in terms of \(x, y\), and \(z\) ) the functional forms of \(L_x \varphi_i^{\prime}, i=1,2,3\).

Hint. Use \(\int e^{-r^2} d^3 r=\pi^{3 / 2}, \int x^2 e^{-r^2} d^3 r=\frac{1}{2} \pi^{3 / 2} ;\) the integrals are over \(\mathbb{R}^3\).

\subsection{Solution}



\newpage
\section{Question 5}

\subsection{Problem}

Using the formal properties of unitary transformations, show that the commutator \([x, p]=i\) is invariant under unitary transformation of the matrices representing \(x\) and \(p\).

\subsection{Solution}



\newpage
\section{Question 6}

\subsection{Problem}

The Pauli matrices
\[
    \sigma_1=\left(\begin{array}{ll}
            0 & 1 \\
            1 & 0
        \end{array}\right), \quad \sigma_2=\left(\begin{array}{cc}
            0 & -i \\
            i & 0
        \end{array}\right), \quad \sigma_3=\left(\begin{array}{rr}
            1 & 0  \\
            0 & -1
        \end{array}\right),
\]
have commutator \(\left[\sigma_1, \sigma_2\right]=2 i \sigma_3\). Show that this relationship continues to be valid if these matrices are transformed by
\[
    \mathbf{U}=\left(\begin{array}{rr}
            \cos \theta  & \sin \theta \\
            -\sin \theta & \cos \theta
        \end{array}\right) .
\]

\subsection{Solution}



\newpage
\bibliographystyle{plain}
\bibliography{references}
\nocite{El-Deeb_PEU-356_Assignments}

\end{document}