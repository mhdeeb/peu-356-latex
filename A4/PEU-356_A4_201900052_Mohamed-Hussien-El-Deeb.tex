\documentclass[12pt]{article}
\usepackage[svgnames,x11names,table]{xcolor}
\usepackage{hyperref}
\usepackage{graphicx}
\usepackage{parskip}
\usepackage{float}
\usepackage{amsmath}
\usepackage{esint}
\usepackage{amssymb}
\usepackage{enumitem}
\usepackage[thicklines]{cancel}

\hypersetup{
    colorlinks,
    citecolor=blue,
    filecolor=black,
    linkcolor=black,
    urlcolor=RoyalBlue4,
}

\title{PEU 356 Assignment 4}
\author{Mohamed Hussien El-Deeb (201900052)}
\date{\today}

\DeclareMathOperator{\sech}{sech}
\DeclareMathOperator{\csch}{csch}

\begin{document}

\maketitle
\tableofcontents
\hypersetup{linkcolor=RoyalBlue4}

\newpage
\section{4.3.7}

\subsection{Problem}

Verify that \(V_{i ; j}=g_{i k} V_{; j}^k\) by showing that

\[
    \frac{\partial V_i}{\partial q^j}-V_k \Gamma_{i j}^k=g_{i k}\left[\frac{\partial V^k}{\partial q^j}+V^m \Gamma_{m j}^k\right] .
\]

\subsection{Solution}

\[
    \frac{\partial V_i}{\partial q^j} = \frac{\partial \left(g_{ik} V^k\right) }{\partial q^j}
    = g_{ik} \frac{\partial V^k}{\partial q^j} + V^k \frac{\partial g_{ik}}{\partial q^j}
\]

\[
    \because \frac{\partial g_{ik}}{\partial q^j} = \varepsilon_{i} \cdot \frac{\partial \varepsilon_k}{\partial q^j} + \varepsilon_k \cdot \frac{\partial \varepsilon_i}{\partial q^j}
\]

\[
    \therefore \frac{\partial V_i}{\partial q^j} = g_{ik} \frac{\partial V^k}{\partial q^j}
    + \left(\varepsilon_i \cdot \frac{\partial \varepsilon_k}{\partial q^j}\right) V^k
    + \left(\varepsilon_k \cdot \frac{\partial \varepsilon_i}{\partial q^j}\right) V^k
\]

\[
    = g_{ik} \frac{\partial V^k}{\partial q^j}
    + \left(\varepsilon^l \cdot \frac{\partial \varepsilon_k}{\partial q^j}\right) V^k g_{il}
    + \left(\varepsilon^k \cdot \frac{\partial \varepsilon_i}{\partial q^j}\right) V_k
\]

\[
    \frac{\partial V_i}{\partial q^j} = g_{ik} \frac{\partial V^k}{\partial q^j} + V^k \Gamma_{k j}^l g_{il} + V_k \Gamma_{i j}^k
\]

\[
    \frac{\partial V_i}{\partial q^j} - V_k \Gamma_{i j}^k = g_{ik} \frac{\partial V^k}{\partial q^j} + V^k \Gamma_{k j}^l g_{il}
\]

\[
    \frac{\partial V_i}{\partial q^j} - V_k \Gamma_{i j}^k
    = g_{ik} \frac{\partial V^k}{\partial q^j}
    + V^l \Gamma_{l j}^k g_{ik}
\]

\[
    \frac{\partial V_i}{\partial q^j}-V_k \Gamma_{i j}^k=g_{i k}\left[\frac{\partial V^k}{\partial q^j}+V^m \Gamma_{m j}^k\right] .
\]

\newpage
\section{4.3.8}

\subsection{Problem}

From the circular cylindrical metric tensor \(g_{i j}\), calculate the \(\Gamma_{i j}^k\) for circular cylindrical coordinates.
Note. There are only three nonvanishing \(\Gamma\).

\subsection{Solution}

\[
    g_{ij} = \left[\begin{array}{ccc}
            1 & 0      & 0 \\
            0 & \rho^2 & 0 \\
            0 & 0      & 1
        \end{array}\right]
\]

\[
    g^{ij} = \left[\begin{array}{ccc}
            1 & 0                & 0 \\
            0 & \frac{1}{\rho^2} & 0 \\
            0 & 0                & 1
        \end{array}\right]
\]

\[
    \Gamma_{i j}^n = \frac{1}{2} g^{n k} \left(\frac{\partial g_{i k}}{\partial q^j} + \frac{\partial g_{j k}}{\partial q^i} - \frac{\partial g_{i j}}{\partial q^k}\right)
\]

\[
    \Gamma_{i j}^n = \frac{1}{2} g^{n n} \left(\frac{\partial g_{i n}}{\partial q^j} + \frac{\partial g_{j n}}{\partial q^i} - \frac{\partial g_{i j}}{\partial q^n}\right)
\]

\[
    \frac{\partial g_{2 2}}{\partial q^1} = \frac{\partial \rho^2}{\partial \rho} = 2\rho
\]

\[
    \Gamma_{2 1}^2 = \Gamma_{1 2}^2 = \frac{1}{\rho}
\]

\[
    \Gamma_{2 2}^1 = -\rho
\]

\newpage
\section{4.3.10}

\subsection{Problem}

Show that for the metric tensor \(g_{ij; k} = g^{ij}_{;k} = 0\).

\subsection{Solution}

\[
    g_{ij;k} = \frac{\partial g_{ij}}{\partial k} - \Gamma_{ik}^\alpha g_{\alpha j} - \Gamma_{jk}^\alpha g_{i\alpha}
\]

\[
    = \frac{\partial g_{ij}}{\partial k} - \frac{1}{2} g_{j\alpha} g^{\alpha \beta} \left(\frac{\partial g_{\beta k}}{\partial i} + \frac{\partial g_{\beta i}}{\partial k} - \frac{\partial g_{ik}}{\partial \beta} \right)
\]

\[
    - \frac{1}{2} g_{i \alpha} g^{\alpha \beta}
    \left(\frac{\partial g_{\beta k}}{\partial j} + \frac{\partial g_{\beta j}}{\partial k} - \frac{\partial g_{jk}}{\partial \beta} \right)
\]

\[
    = \frac{\partial g_{ij}}{\partial k} - \frac{1}{2} \left( \frac{\partial g_{jk}}{\partial i} + \frac{\partial g_{ji}}{\partial k} - \frac{\partial g_{ik}}{\partial j} \right) - \frac{1}{2} \left( \frac{\partial g_{ik}}{\partial j} + \frac{\partial g_{ij}}{\partial k} - \frac{\partial g_{jk}}{\partial i} \right) = 0
\]

Contravariant g is just a function of covariant g, so the same applies for \(g^{ij}_{;k}\).
if covariant g does not depend on k, then neither does contravariant g.

\newpage
\section{4.3.12}

\subsection{Problem}

The covariant vector \(A_i\) is the gradient of a scalar. Show that the difference of covariant
derivatives \(A_{i;j} - A_{j;i}\) vanishes.

\subsection{Solution}

\[
    A_{i ; j} = \frac{\partial A_i}{\partial q^j} - A_k \Gamma_{i j}^k
\]

\[
    A_i = \frac{\partial \phi}{\partial q^i}
\]

\[
    A_{i ; j} = \frac{\partial^2 \phi}{\partial q^j \partial q^i} - \frac{\partial \phi}{\partial q^k} \Gamma_{i j}^k
\]

\[
    A_{j ; i} = \frac{\partial^2 \phi}{\partial q^i \partial q^j} - \frac{\partial \phi}{\partial q^k} \Gamma_{j i}^k
\]

Since the partial derivatives commute and the Christoffel symbols are symmetric in their lower indices, \(A_{i ; j} - A_{j ; i} = 0\).

\newpage
\section{4.4.1}

\subsection{Problem}

Assuming the functions \(u\) and \(v\) to be differentiable,

(a) Show that a necessary and sufficient condition that \(u(x, y, z)\) and \(v(x, y, z)\) are related by some function \(f(u, v)=0\) is that \((\nabla u) \times(\nabla v)=0\);

(b) If \(u=u(x, y)\) and \(v=v(x, y)\), show that the condition \((\nabla u) \times(\nabla v)=0\) leads to the 2-D Jacobian

\[
    J=\frac{\partial(u, v)}{\partial(x, y)}=\left|\begin{array}{ll}
        \frac{\partial u}{\partial x} & \frac{\partial u}{\partial y} \\
        \frac{\partial v}{\partial x} & \frac{\partial v}{\partial y}
    \end{array}\right|=0 .
\]

\subsection{Solution}

\subsubsection{Part (a)}

\[
    f = 0
\]

\[
    \nabla f = \frac{\partial f}{\partial u} \nabla u + \frac{\partial f}{\partial v} \nabla v = 0
\]

\[
    \frac{\partial f}{\partial u} \nabla u = -\frac{\partial f}{\partial v} \nabla v
\]

\[
    \nabla u = c \nabla v
\]

\[
    (\nabla u) \times (\nabla v) = c (\nabla v) \times (\nabla v) = 0
\]

\subsubsection{Part (b)}

\[
    (\nabla u) \times (\nabla v) = \left|\begin{array}{ccc}
        \hat{\imath}                  & \hat{\jmath}                  & \hat{k} \\
        \frac{\partial u}{\partial x} & \frac{\partial u}{\partial y} & 0       \\
        \frac{\partial v}{\partial x} & \frac{\partial v}{\partial y} & 0
    \end{array}\right|
\]

\[
    \implies \frac{\partial u}{\partial x} \frac{\partial v}{\partial y} = \frac{\partial u}{\partial y} \frac{\partial v}{\partial x}
\]

\[
    \frac{\partial (u,v)}{\partial (x,y)} = \begin{vmatrix}
        \frac{\partial u}{\partial x} & \frac{\partial u}{\partial y} \\
        \frac{\partial v}{\partial x} & \frac{\partial v}{\partial y}
    \end{vmatrix} = 0
\]

\[
    \implies \frac{\partial u}{\partial x} \frac{\partial v}{\partial y} = \frac{\partial u}{\partial y} \frac{\partial v}{\partial x}
\]

\newpage
\section{4.4.2}

\subsection{Problem}

A 2-D orthogonal system is described by the coordinates \(q_{1}\) and \(q_{2}\). Show that the Jacobian \(J\) satisfies the equation

\[
    J \equiv \frac{\partial(x, y)}{\partial\left(q_{1}, q_{2}\right)} \equiv \frac{\partial x}{\partial q_{1}} \frac{\partial y}{\partial q_{2}}-\frac{\partial x}{\partial q_{2}} \frac{\partial y}{\partial q_{1}}=h_{1} h_{2} .
\]

Hint. It's easier to work with the square of each side of this equation.

\subsection{Solution}

\[
    {\left(\frac{\partial x}{\partial q_{1}} \frac{\partial y}{\partial q_{2}}-\frac{\partial x}{\partial q_{2}} \frac{\partial y}{\partial q_{1}}\right) }^2
    = {\left(\frac{\partial x}{\partial q_{1}} \frac{\partial y}{\partial q_{2}}\right)}^2
    + {\left(\frac{\partial x}{\partial q_{2}} \frac{\partial y}{\partial q_{1}}\right)}^2
    - 2 \frac{\partial x}{\partial q_{1}} \frac{\partial x}{\partial q_{2}} \frac{\partial y}{\partial q_{2}} \frac{\partial y}{\partial q_{1}}
\]

\[
    {h_i}^2 = {\left(\frac{\partial x}{\partial q_i}\right)}^2  + {\left(\frac{\partial y}{\partial q_i}\right)}^2
\]

\[
    {h_1}^2{h_2}^2 = \left({\left(\frac{\partial x}{\partial q_1}\right)}^2  + {\left(\frac{\partial y}{\partial q_1}\right)}^2\right)\left({\left(\frac{\partial x}{\partial q_2}\right)}^2  + {\left(\frac{\partial y}{\partial q_2}\right)}^2\right)
\]

\[
    = {\left(\frac{\partial x}{\partial q_1} \frac{\partial y}{\partial q_2}\right)}^2
    + {\left(\frac{\partial x}{\partial q_2} \frac{\partial y}{\partial q_1}\right)}^2
    - 2 \frac{\partial x}{\partial q_{1}} \frac{\partial x}{\partial q_{2}} \frac{\partial y}{\partial q_{2}} \frac{\partial y}{\partial q_{1}}
\]

\[
    + {\left(\frac{\partial x}{\partial q_1} \frac{\partial x}{\partial q_2}\right)}^2
    + {\left(\frac{\partial y}{\partial q_1} \frac{\partial y}{\partial q_2}\right)}^2
    + 2 \frac{\partial x}{\partial q_{1}} \frac{\partial x}{\partial q_{2}} \frac{\partial y}{\partial q_{2}} \frac{\partial y}{\partial q_{1}}
\]

\[
    = {\left(\frac{\partial x}{\partial q_{1}} \frac{\partial y}{\partial q_{2}}-\frac{\partial x}{\partial q_{2}} \frac{\partial y}{\partial q_{1}}\right) }^2
    + {\left(\frac{\partial x}{\partial q_{1}} \frac{\partial x}{\partial q_{2}}+\frac{\partial y}{\partial q_{1}} \frac{\partial y}{\partial q_{2}}\right) }^2
\]

\[
    \frac{\partial x}{\partial q_{1}} \frac{\partial x}{\partial q_{2}}+\frac{\partial y}{\partial q_{1}} \frac{\partial y}{\partial q_{2}}
    = \varepsilon_1 \cdot \varepsilon_2 = 0
\]

\[
    {h_1}^2{h_2}^2 = {\left(\frac{\partial x}{\partial q_{1}} \frac{\partial y}{\partial q_{2}}-\frac{\partial x}{\partial q_{2}} \frac{\partial y}{\partial q_{1}}\right) }^2
\]

\[
    h_1h_2 = \frac{\partial x}{\partial q_{1}} \frac{\partial y}{\partial q_{2}}-\frac{\partial x}{\partial q_{2}} \frac{\partial y}{\partial q_{1}}
\]

\newpage
\section{4.4.3}

\subsection{Problem}

For the transformation \(u=x+y, v=x / y\), with \(x \geq 0\) and \(y \geq 0\), find the Jacobian \(\frac{\partial(x, y)}{\partial(u, v)}\)

(a) By direct computation,

(b) By first computing \(J^{-1}\).

\subsection{Solution}

\subsubsection{Part (a)}

\[
    x = \frac{uv}{1 + v} \quad y = \frac{u}{1 + v}
\]
\[
    \frac{\partial x}{\partial u} = \frac{v}{1 + v} \quad \frac{\partial x}{\partial v} = \frac{u}{{(1 + v)}^2}
\]

\[
    \frac{\partial y}{\partial u} = \frac{1}{1 + v} \quad \frac{\partial y}{\partial v} = -\frac{u}{{(1 + v)}^2}
\]

\[
    \frac{\partial(x, y)}{\partial(u, v)} = \begin{vmatrix}
        \frac{v}{1 + v} & \frac{u}{{(1 + v)}^2}  \\
        \frac{1}{1 + v} & -\frac{u}{{(1 + v)}^2}
    \end{vmatrix} = - \frac{uv}{{(1 + v)}^3} - \frac{u}{{(1 + v)}^3} = -\frac{u}{{(1 + v)}^2}
\]

\subsubsection{Part (b)}

\[
    J^{-1} = \frac{\partial(u, v)}{\partial(x, y)} = \begin{vmatrix}
        1   & 1      \\
        1/y & -x/y^2
    \end{vmatrix} = - \frac{x}{y^2} - \frac{1}{y} = -\frac{x + y}{y^2} = -\frac{u}{\frac{u^2}{{(1+v)}^2}}
\]

\[
    J^{-1}  = -\frac{{(1+v)}^2}{u}
\]

\[
    J = -\frac{u}{{(1+v)}^2}
\]

\newpage
\bibliographystyle{plain}
\bibliography{references}
\nocite{arfken2013mathematical}
\nocite{El-Deeb_PEU-356_Assignments}

\end{document}