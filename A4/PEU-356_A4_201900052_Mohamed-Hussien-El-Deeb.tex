\documentclass[12pt]{article}
\usepackage[svgnames,x11names,table]{xcolor}
\usepackage{hyperref}
\usepackage{graphicx}
\usepackage{parskip}
\usepackage{float}
\usepackage{amsmath}
\usepackage{esint}
\usepackage{amssymb}
\usepackage{enumitem}
\usepackage[thicklines]{cancel}

\hypersetup{
    colorlinks,
    citecolor=blue,
    filecolor=black,
    linkcolor=black,
    urlcolor=RoyalBlue4,
}

\title{PEU 356 Assignment 4}
\author{Mohamed Hussien El-Deeb (201900052)}
\date{\today}

\DeclareMathOperator{\sech}{sech}
\DeclareMathOperator{\csch}{csch}

\begin{document}

\maketitle
\tableofcontents
\hypersetup{linkcolor=RoyalBlue4}

\newpage
\section{4.3.7}

\subsection{Problem}

Verify that \(V_{i ; j}=g_{i k} V_{; j}^k\) by showing that

\[
    \frac{\partial V_i}{\partial q^j}-V_k \Gamma_{i j}^k=g_{i k}\left[\frac{\partial V^k}{\partial q^j}+V^m \Gamma_{m j}^k\right] .
\]

\subsection{Solution}

\newpage
\section{4.3.8}

\subsection{Problem}

From the circular cylindrical metric tensor \(g_{i j}\), calculate the \(\Gamma_{i j}^k\) for circular cylindrical coordinates.
Note. There are only three nonvanishing \(\Gamma\).

\subsection{Solution}

\newpage
\section{4.3.10}

\subsection{Problem}

Show that for the metric tensor \(g_{ij; k} = g^{ij}_{;k} = 0\).

\subsection{Solution}

\newpage
\section{4.3.12}

\subsection{Problem}

The covariant vector \(A_i\) is the gradient of a scalar. Show that the difference of covariant
derivatives \(A_{i;j} - A_{j;i}\) vanishes.

\subsection{Solution}

\newpage
\section{4.4.1}

\subsection{Problem}

Assuming the functions \(u\) and \(v\) to be differentiable,

(a) Show that a necessary and sufficient condition that \(u(x, y, z)\) and \(v(x, y, z)\) are related by some function \(f(u, v)=0\) is that \((\nabla u) \times(\nabla v)=0\);

(b) If \(u=u(x, y)\) and \(v=v(x, y)\), show that the condition \((\nabla u) \times(\nabla v)=0\) leads to the 2-D Jacobian

\[
J=\frac{\partial(u, v)}{\partial(x, y)}=\left|\begin{array}{ll}
\frac{\partial u}{\partial x} & \frac{\partial u}{\partial y} \\
\frac{\partial v}{\partial x} & \frac{\partial v}{\partial y}
\end{array}\right|=0 .
\]

\subsection{Solution}

\newpage
\section{4.4.2}

\subsection{Problem}

A 2-D orthogonal system is described by the coordinates \(q_{1}\) and \(q_{2}\). Show that the Jacobian \(J\) satisfies the equation

\[
J \equiv \frac{\partial(x, y)}{\partial\left(q_{1}, q_{2}\right)} \equiv \frac{\partial x}{\partial q_{1}} \frac{\partial y}{\partial q_{2}}-\frac{\partial x}{\partial q_{2}} \frac{\partial y}{\partial q_{1}}=h_{1} h_{2} .
\]

Hint. It's easier to work with the square of each side of this equation.

\subsection{Solution}

\newpage
\section{4.4.3}

\subsection{Problem}

For the transformation \(u=x+y, v=x / y\), with \(x \geq 0\) and \(y \geq 0\), find the Jacobian \(\frac{\partial(x, y)}{\partial(u, v)}\)

(a) By direct computation,

(b) By first computing \(J^{-1}\).

\subsection{Solution}

\newpage
\bibliographystyle{plain}
\bibliography{references}
\nocite{El-Deeb_PEU-356_Assignments}

\end{document}