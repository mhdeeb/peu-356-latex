\documentclass[12pt]{article}
\usepackage[svgnames,x11names,table]{xcolor}
\usepackage{hyperref}
\usepackage{graphicx}
\usepackage{parskip}
\usepackage{float}
\usepackage{amsmath}
\usepackage{esint}
\usepackage{amssymb}
\usepackage{enumitem}
\usepackage[thicklines]{cancel}

\hypersetup{
    colorlinks,
    citecolor=blue,
    filecolor=black,
    linkcolor=black,
    urlcolor=RoyalBlue4,
}

\title{PEU 356 Assignment 7}
\author{Mohamed Hussien El-Deeb (201900052)}
\date{\today}

\DeclareMathOperator{\sech}{sech}
\DeclareMathOperator{\csch}{csch}

\begin{document}

\maketitle
\tableofcontents
\hypersetup{linkcolor=RoyalBlue4}

\newpage
\section{Question 1}

\subsection{Problem}

Find the eigenvalues and corresponding normalized eigenvectors of the matrix

\[
    A = \begin{pmatrix}
        0 & 1 & 1 \\
        1 & 0 & 1 \\
        1 & 1 & 0 \\
    \end{pmatrix}
\]

Orthogonalize any degenerate eigenvectors.

ANS. \(\lambda = -1, -1, 2. \)

\subsection{Solution}

det(\(A - \lambda I\)) = 0

\[
    \begin{vmatrix}
        -\lambda & 1        & 1        \\
        1        & -\lambda & 1        \\
        1        & 1        & -\lambda \\
    \end{vmatrix} = 0
\]

\[
    \begin{vmatrix}
        \lambda^2 & -\lambda \\
        1         & -\lambda \\
    \end{vmatrix} + \begin{vmatrix}
        1   & 1       \\
        - 1 & \lambda \\
    \end{vmatrix} + \begin{vmatrix}
        1 & -\lambda \\
        1 & 1        \\
    \end{vmatrix} = 0
\]

\[
    \lambda^3 - 3 \lambda - 2 = 0
\]

\[
    (\lambda + 1)(\lambda + 1)(\lambda - 2) = 0
\]

\[
    \lambda = -1, -1, 2
\]

For \(\lambda = -1\)

\[
    A\vec{v} = -\vec{v}
\]

\[
    \begin{pmatrix}
        0 & 1 & 1 \\
        1 & 0 & 1 \\
        1 & 1 & 0 \\
    \end{pmatrix} \begin{pmatrix}
        x \\
        y \\
        z \\
    \end{pmatrix} = \begin{pmatrix}
        -x \\
        -y \\
        -z \\
    \end{pmatrix}
\]

\[
    y + z = -x
\]

\[
    x + z = -y
\]

\[
    x + y = -z
\]

\[
    \vec{v} = \begin{pmatrix}
        x    \\
        y    \\
        -x-y \\
    \end{pmatrix}
\]

\[
    \vec{v} = x \begin{pmatrix}
        1  \\
        0  \\
        -1 \\
    \end{pmatrix} + y \begin{pmatrix}
        0  \\
        1  \\
        -1 \\
    \end{pmatrix}
\]

\[
    \vec{v_1} = \frac{1}{\sqrt{2}} \begin{pmatrix}
        1  \\
        0  \\
        -1 \\
    \end{pmatrix},\quad \vec{v_2} = \frac{1}{\sqrt{2}} \begin{pmatrix}
        0  \\
        1  \\
        -1 \\
    \end{pmatrix}
\]

For \(\lambda = 2\)

\[
    A\vec{v} = 2\vec{v}
\]

\[
    \begin{pmatrix}
        0 & 1 & 1 \\
        1 & 0 & 1 \\
        1 & 1 & 0 \\
    \end{pmatrix} \begin{pmatrix}
        x \\
        y \\
        z \\
    \end{pmatrix} = \begin{pmatrix}
        2x \\
        2y \\
        2z \\
    \end{pmatrix}
\]

\[
    y + z = 2x
\]

\[
    x + z = 2y
\]

\[
    x + y = 2z
\]

\[
    y = z = x
\]

\[
    \vec{v} = x \begin{pmatrix}
        1 \\
        1 \\
        1 \\
    \end{pmatrix}
\]

\[
    \vec{v_3} = \frac{1}{\sqrt{3}} \begin{pmatrix}
        1 \\
        1 \\
        1 \\
    \end{pmatrix}
\]

\newpage
\section{Question 2}

\subsection{Problem}

Find the eigenvalues and corresponding normalized eigenvectors of the matrix

\[
    A = \begin{pmatrix}
        5 & 0 & 2 \\
        0 & 1 & 0 \\
        2 & 0 & 2 \\
    \end{pmatrix}
\]

Orthogonalize any degenerate eigenvectors.

ANS. \(\lambda = 1, 1, 6. \)

\subsection{Solution}

det(\(A - \lambda I\)) = 0

\[
    \begin{vmatrix}
        5 - \lambda & 0           & 2           \\
        0           & 1 - \lambda & 0           \\
        2           & 0           & 2 - \lambda \\
    \end{vmatrix} = 0
\]

\[
    (5 - \lambda)(1 - \lambda)(2 - \lambda) - 4(1 - \lambda) = 0
\]

\[
    (1 - \lambda)((5 - \lambda)(2 - \lambda) - 4) = 0
\]

\[
    \lambda_1 = 1
\]

\[
    (5 - \lambda)(2 - \lambda) = 4
\]

\[
    \lambda_2 = 1, \quad \lambda_3 = 6
\]

For \(\lambda = 1\)

\[
    A\vec{v} = \vec{v}
\]

\[
    \begin{pmatrix}
        5 & 0 & 2 \\
        0 & 1 & 0 \\
        2 & 0 & 2 \\
    \end{pmatrix} \begin{pmatrix}
        x \\
        y \\
        z \\
    \end{pmatrix} = \begin{pmatrix}
        x \\
        y \\
        z \\
    \end{pmatrix}
\]

\[
    y = y
\]

\[
    z = -2x
\]


\[
    \vec{v} = \begin{pmatrix}
        x   \\
        y   \\
        -2x \\
    \end{pmatrix}
\]

\[
    \vec{v} = x \begin{pmatrix}
        1  \\
        0  \\
        -2 \\
    \end{pmatrix} +
    y \begin{pmatrix}
        0 \\
        1 \\
        0 \\
    \end{pmatrix}
\]

\[
    \vec{v_1} = \frac{1}{\sqrt{5}} \begin{pmatrix}
        1  \\
        0  \\
        -2 \\
    \end{pmatrix},\quad \vec{v_2} = \begin{pmatrix}
        0 \\
        1 \\
        0 \\
    \end{pmatrix}
\]

For \(\lambda = 6\)

\[
    A\vec{v} = 6\vec{v}
\]

\[
    \begin{pmatrix}
        5 & 0 & 2 \\
        0 & 1 & 0 \\
        2 & 0 & 2 \\
    \end{pmatrix} \begin{pmatrix}
        x \\
        y \\
        z \\
    \end{pmatrix} = \begin{pmatrix}
        6x \\
        6y \\
        6z \\
    \end{pmatrix}
\]

\[
    2z = x
\]

\[
    y = y
\]

\[
    \vec{v} = \begin{pmatrix}
        2z \\
        y  \\
        z  \\
    \end{pmatrix}
\]

\[
    \vec{v} = z \begin{pmatrix}
        2 \\
        0 \\
        1 \\
    \end{pmatrix} +
    y \begin{pmatrix}
        0 \\
        1 \\
        0 \\
    \end{pmatrix}
\]

\[
    \vec{v_3} = \frac{1}{\sqrt{5}} \begin{pmatrix}
        2 \\
        0 \\
        1 \\
    \end{pmatrix}
\]

\newpage
\section{Question 3}

\subsection{Problem}

Describe the geometric properties of the surface

\[
    x^2+2 x y+2 y^2+2 y z+z^2=1 .
\]

How is it oriented in 3-D space? Is it a conic section? If so, which kind?

\subsection{Solution}

Writing the quadratic form equation in matrix form, we get,

\[
    A = \begin{pmatrix}
        1 & 1 & 0 \\
        1 & 2 & 1 \\
        0 & 1 & 1 \\
    \end{pmatrix}
\]

\[
    \begin{vmatrix}
        1 - \lambda & 1           & 0           \\
        1           & 2 - \lambda & 1           \\
        0           & 1           & 1 - \lambda \\
    \end{vmatrix} = 0
\]

\[
    (1 - \lambda)((2 - \lambda)(1 - \lambda) - 2) = 0
\]

\[
    \lambda_1 = 1
\]

\[
    (2 - \lambda)(1 - \lambda) = 2
\]

\[
    \lambda_2 = 3, \quad \lambda_3 = 0
\]

\[
    a^2 + 3b^2 = 1
\]

Since we have two positive eigenvalues and one zero eigenvalue, the surface is an elliptic cylinder.

The direction is the eigenvector corresponding to the zero eigenvalue.

\[
    \begin{pmatrix}
        1 & 1 & 0 \\
        1 & 2 & 1 \\
        0 & 1 & 1 \\
    \end{pmatrix} \begin{pmatrix}
        x \\
        y \\
        z \\
    \end{pmatrix} = \begin{pmatrix}
        0 \\
        0 \\
        0 \\
    \end{pmatrix}
\]

\[
    x = -y
\]

\[
    x = z
\]

\[
    \vec{v} = x \begin{pmatrix}
        1  \\
        -1 \\
        1  \\
    \end{pmatrix}
\]

\[
    \vec{v_3} = \frac{1}{\sqrt{3}} \begin{pmatrix}
        1  \\
        -1 \\
        1  \\
    \end{pmatrix}
\]

Note that this is not an equation of a conic section, but rather would produce a conic section if a plane with normal vector \(\vec{v_3}\) and passes by the origin intersects the surface.

To get the transformation matrix we need to find the other two normalized eigenvectors.

For \(\lambda = 1\)

\[
    \begin{pmatrix}
        1 & 1 & 0 \\
        1 & 2 & 1 \\
        0 & 1 & 1 \\
    \end{pmatrix} \begin{pmatrix}
        x \\
        y \\
        z \\
    \end{pmatrix} = \begin{pmatrix}
        x \\
        y \\
        z \\
    \end{pmatrix}
\]

\[
    y = 0
\]

\[
    x + z = 0
\]

\[
    \vec{v} = x \begin{pmatrix}
        1  \\
        0  \\
        -1 \\
    \end{pmatrix}
\]

\[
    \vec{v_1} = \frac{1}{\sqrt{2}} \begin{pmatrix}
        1  \\
        0  \\
        -1 \\
    \end{pmatrix}
\]

For \(\lambda = 3\)

\[
    \begin{pmatrix}
        1 & 1 & 0 \\
        1 & 2 & 1 \\
        0 & 1 & 1 \\
    \end{pmatrix} \begin{pmatrix}
        x \\
        y \\
        z \\
    \end{pmatrix} = \begin{pmatrix}
        3x \\
        3y \\
        3z \\
    \end{pmatrix}
\]

\[
    y = 2x
\]

\[
    z = x
\]

\[
    \vec{v} = x \begin{pmatrix}
        1 \\
        2 \\
        1 \\
    \end{pmatrix}
\]

\[
    \vec{v_2} = \frac{1}{\sqrt{6}} \begin{pmatrix}
        1 \\
        2 \\
        1 \\
    \end{pmatrix}
\]

\[
    T = \begin{pmatrix}
        1/\sqrt{2}  & 1/\sqrt{6} & 1/\sqrt{3}  \\
        0           & 2/\sqrt{6} & -1/\sqrt{3} \\
        -1/\sqrt{2} & 1/\sqrt{6} & 1/\sqrt{3}  \\
    \end{pmatrix}
\]

\newpage
\section{Question 4}

\subsection{Problem}

A system of \(N\) degrees of freedoms is subject to the following potential

\[
    V=\frac{1}{2} V_{i j} x_i x_j, \quad i, j=1, \ldots, N .
\]

(i) Derive Newton's second law for the system.

(ii) Assume we are looking for solutions of the form

\[
    X(t)=X(0) \sin \omega t, \quad X(t)=\left(\begin{array}{c}
            x_1(t) \\
            \vdots \\
            x_N(t)
        \end{array}\right) .
\]

Show that \(X(0)\) satisfies an eigenvalue problem.

(iii) What are the conditions on the eigenvalues of \(V_{i j}\) in order to have normal modes (i.e., oscillatory SII motion with a fixed frequency).

\subsection{Solution}

(i) Newton's second law for the system is given by

\[
    {\left(m_k \ddot{x}_k\right) }_k = -\frac{\partial V}{\partial x_k}
\]

\[
    {\left(m_k \ddot{x}_k\right) }_k = -\frac{1}{2} V_{ij} \frac{\partial}{\partial x_k} (x_i x_j)
\]

\[
    {\left(m_k \ddot{x}_k\right) }_k = -\frac{1}{2} V_{ij} \left( \delta_{ik} x_j + \delta_{jk} x_i \right)
\]

\[
    {\left(m_i \ddot{x}_i\right) }_i = -\frac{1}{2} \left(V_{ij} + V_{ji}\right) x_i
    \ ||\ {\left(m_i \ddot{x}_i\right) }_i = -\frac{1}{2} {\left(\left(V + V^T\right) \vec{x}\right) }_i
\]

(ii)

\[
    \overrightarrow{x(t)} = \sin{\left(\omega t\right)} \overrightarrow{x_0}
\]

\[
    \overrightarrow{\ddot{x}(t)} = -\omega^2\sin{\left(\omega t\right)} \overrightarrow{x_0}
\]

\[
    {\left(m_i \ddot{x}_i\right) }_i = -\frac{1}{2} \left(V_{ij} + V_{ji}\right) x_i
\]

\[
    \omega^2 m_i x_i = \frac{1}{2} \left(V_{ij} + V_{ji}\right) x_i
\]

\[
    \omega^2 m_i = \frac{1}{2} \left(V_{ij} + V_{ji}\right)
\]

\[
    \omega = \pm \sqrt{\frac{\sum_j V_{0j} + V_{j0}}{2 m_0}}
\]

If we assume m is a constant and V is symmetric, then

\[
    \omega^2 m \vec{x} = V \vec{x}
\]

Therefore, \(\vec{x}\) satisfies an eigenvalue problem.

\[
    \omega = \pm \frac{1}{\sqrt{m}} \sqrt{\sum_j V_{0j}}
\]

(iii)

The eigenvalues of \(V_{ij}\) must be positive in order to have normal modes.

\[
    \omega^2 \geq 0
\]


\newpage
\section{Question 5}

\subsection{Problem}

Consider the previous example with the following potential

\[
    V=x_1 x_2+x_1 x_3+x_2 x_3 .
\]

Find the normal modes.

\subsection{Solution}

\[
    V = \begin{pmatrix}
        0           & \frac{1}{2} & \frac{1}{2} \\
        \frac{1}{2} & 0           & \frac{1}{2} \\
        \frac{1}{2} & \frac{1}{2} & 0           \\
    \end{pmatrix}
\]

\[
    \begin{vmatrix}
        -\lambda    & \frac{1}{2} & \frac{1}{2} \\
        \frac{1}{2} & -\lambda    & \frac{1}{2} \\
        \frac{1}{2} & \frac{1}{2} & -\lambda    \\
    \end{vmatrix} = 0
\]

\[
    \lambda \left( \lambda^2 - \frac{1}{4} \right)
    - \frac{1}{2} \left( \frac{\lambda}{2} + \frac{1}{4} \right)
    - \frac{1}{2} \left( \frac{\lambda}{2} + \frac{1}{4} \right) = 0
\]

\[
    \left(\lambda-1\right) {\left(2\lambda+1\right) }^2 = 0
\]

\[
    \lambda = 1, -\frac{1}{2}, -\frac{1}{2}
\]

For \(\lambda = 1\)

\[
    \begin{pmatrix}
        0           & \frac{1}{2} & \frac{1}{2} \\
        \frac{1}{2} & 0           & \frac{1}{2} \\
        \frac{1}{2} & \frac{1}{2} & 0           \\
    \end{pmatrix} \begin{pmatrix}
        x \\
        y \\
        z \\
    \end{pmatrix} = \begin{pmatrix}
        x \\
        y \\
        z \\
    \end{pmatrix}
\]

\[
    y + z = 2x
\]

\[
    x + z = 2y
\]

\[
    x + y = 2z
\]
\[
    x = y = z
\]

\[
    \vec{v} = x \begin{pmatrix}
        1 \\
        1 \\
        1 \\
    \end{pmatrix}
\]

\[
    \vec{v_1} = \frac{1}{\sqrt{3}} \begin{pmatrix}
        1 \\
        1 \\
        1 \\
    \end{pmatrix}
\]

For \(\lambda = -\frac{1}{2}\)

\[
    \begin{pmatrix}
        0           & \frac{1}{2} & \frac{1}{2} \\
        \frac{1}{2} & 0           & \frac{1}{2} \\
        \frac{1}{2} & \frac{1}{2} & 0           \\
    \end{pmatrix} \begin{pmatrix}
        x \\
        y \\
        z \\
    \end{pmatrix} = -\frac{1}{2} \begin{pmatrix}
        x \\
        y \\
        z \\
    \end{pmatrix}
\]

\[
    x + y = - z
\]

\[
    \vec{v} = \begin{pmatrix}
        x      \\
        y      \\
        -x - y \\
    \end{pmatrix}
\]

\[
    \vec{v} = x \begin{pmatrix}
        1 \\
        0 \\
        -1 \\
    \end{pmatrix} +
    y \begin{pmatrix}
        0 \\
        1 \\
        -1 \\
    \end{pmatrix}
\]

\[
    \vec{v_2} = \frac{1}{\sqrt{2}} \begin{pmatrix}
        1 \\
        0 \\
        -1 \\
    \end{pmatrix},\quad \vec{v_3} = \frac{1}{\sqrt{2}} \begin{pmatrix}
        0 \\
        1 \\
        -1 \\
    \end{pmatrix}
\]



\newpage
\bibliographystyle{plain}
\bibliography{references}
\nocite{El-Deeb_PEU-356_Assignments}

\end{document}